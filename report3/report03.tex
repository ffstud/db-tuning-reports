\documentclass[11pt]{scrartcl}

\usepackage[top=1.5cm]{geometry}
\usepackage{float}
\usepackage{listings}
\usepackage{xcolor}
\usepackage{csquotes}

\usepackage[
  backend=biber,
  style=numeric,
  sorting=none
]{biblatex}

\addbibresource{references/references.bib}

\setlength{\parindent}{0em}
\setlength{\parskip}{0.5em}

\newcommand{\youranswerhere}{[Your answer goes here \ldots]}
\renewcommand{\thesubsection}{\arabic{subsection}}

\lstdefinestyle{dbtsql}{
  language=SQL,
  basicstyle=\small\ttfamily,
  keywordstyle=\color{magenta!75!black},
  stringstyle=\color{green!50!black},
  showspaces=false,
  showstringspaces=false,
  commentstyle=\color{gray}}

\title{
  \textbf{\large Assignment 3} \\
  Index Tuning \\
  {\large Database Tuning}}

\author{
  New Group 8 \\
  \large Frauenschuh Florian, 12109584 \\
  \large Lindner Peter, 12101607 \\
  \large Weilert Alexander, 12119653
}

\begin{document}

\maketitle

\textbf{Database system and version:} \texttt{Postgres 14.11} with driver \texttt{postgresql 42.7.3} \\

\subsection{Index Data Structures}

Which index data structures (e.g., B$^+$ tree index) are supported?

\begin{itemize}
  \item B-Tree \cite{PostgreSQL2024IndexTypes}
  \begin{itemize}
    \item Default when using \texttt{CREATE INDEX}
    \item   \begin{lstlisting}[style=dbtsql]
CREATE INDEX index_name ON table_name (column_name);
    \end{lstlisting}
  \end{itemize}

  \item Hash \cite{PostgreSQL2024IndexTypes}
  \begin{itemize}
    \item Hash index stores 32-bit hash derived from indexed column
    \item Only supports equality comparisons
    \item  \begin{lstlisting}[style=dbtsql]
CREATE INDEX index_name ON table_name USING HASH (column_name);
             \end{lstlisting}
  \end{itemize}
  \item GiST \cite{PostgreSQL2024IndexTypes} \cite{PostgreSQL2024GistGin}
  \begin{itemize}
    \item \enquote{Generalized Search Tree}
    \item Infrastructure of many index strategies
    \item Lossy index (may produce false matches)
    \item \begin{lstlisting}[style=dbtsql]
CREATE INDEX index_name ON table_name USING gist(column_name);
    \end{lstlisting}
  \end{itemize}
  \item SP-GiST \cite{PostgreSQL2024IndexTypes}
  \begin{itemize}
    \item \enquote{Space-partitioned GiST}
    \item Non-balanced data structures
    \item \begin{lstlisting}[style=dbtsql]
CREATE INDEX index_name ON table_name USING spgist(column_name);
\end{lstlisting}
  \end{itemize}
  \item GIN \cite{PostgreSQL2024IndexTypes} \cite{PostgreSQL2024GistGin}
  \begin{itemize}
    \item \enquote{Generalized Inverted Index}
    \item Supports many different indexing strategies
    \item Appropriate for values containing multiple component, e.g.: arrays
    \begin{lstlisting}[style=dbtsql]
CREATE INDEX index_name ON table_name USING gin(column_name);
    \end{lstlisting}
  \end{itemize}
  \item BRIN \cite{PostgreSQL2024IndexTypes}
  \begin{itemize}
    \item \enquote{Block Range INdex}
    \item Indexes ranges of values
    \item \begin{lstlisting}[style=dbtsql]
CREATE INDEX index_name ON table_name USING brin(column_name);
    \end{lstlisting}
  \end{itemize}
\end{itemize}

\subsection{Clustering Indexes}

Discuss how the system supports clustering indexes, in particular:

\paragraph{a)}

How do you create a clustering index on \texttt{ssnum}? Show the query.\footnote{Give the queries for creating a hash index \emph{and} a B$^+$ tree index if both of them are supported.}

First, we assume the table \texttt{Employee} to be the same as in the previous assignment:
\begin{lstlisting}[style=dbtsql]
CREATE TABLE IF NOT EXISTS Employee (
    ssnum INTEGER PRIMARY KEY,
    name VARCHAR(64) UNIQUE NOT NULL,
    manager VARCHAR(64),
    dept VARCHAR(64),
    salary INTEGER,
    numfriends INTEGER);
\end{lstlisting}

We note that \texttt{ssnum} is the primary key of the table, hence Postgres automatically creates an B-Tree based,
\textit{unique index} for it \cite{PostgreSQL2024Unique}.

Now to be able to create a clustering index, we need an index to begin with.
As mentioned, such an index already exists for \texttt{ssnum}.
Thus, we can \textit{cluster} this index according to \cite{PostgreSQL2024Cluster} by performing:

\begin{lstlisting}[style=dbtsql]
CLUSTER Employee USING idx_ssnum;
\end{lstlisting}

Here we assumed the index on \texttt{ssnum} to be named \texttt{idx\_ssnum}.

If we had to create such an index by ourselves, one could accomplish this by performing:
\begin{lstlisting}[style=dbtsql]
CREATE UNIQUE INDEX idx_ssnum ON Employee [USING BTREE] (ssnum);
\end{lstlisting}

The additional command in parentheses \texttt{USING BTREE} is optional, since it already is the default for Postgres
(and Postgres only supports unique indexes using B-Trees) \cite{PostgreSQL2024Unique}.

Now in order to create a clustered hash index, we first need to create the index itself following
\cite{PostgreSQL2024IndexTypes}:

\begin{lstlisting}[style=dbtsql]
CREATE INDEX idx_ssnum ON Employee USING HASH (ssnum);
\end{lstlisting}

We note that we now can no longer create a unique index, since it is not supported as mentioned before.
Afterwards, we again can cluster this created index:

\begin{lstlisting}[style=dbtsql]
CLUSTER Employee USING idx_ssnum;
\end{lstlisting}

\paragraph{b)}

Are clustering indexes on non-key attributes supported, e.g., on \texttt{name}? Show the query.

Yes, clustering indexes are also supported on non-key attributes.
As mentioned earlier, we first need an index on name to cluster it:

\begin{lstlisting}[style=dbtsql]
CREATE INDEX idx_name on Employee (name);
\end{lstlisting}

Followed by that, we can now cluster this created index by performing:

\begin{lstlisting}[style=dbtsql]
CLUSTER Employee USING idx_name;
\end{lstlisting}

\paragraph{c)}

Is the clustering index dense or sparse?

In general, Postgres only supports dense indexes \cite{SO_PostgresDenseSpare}.
Hence, clustering indexes are dense as well.
The only exception from this is the GIN index type, which is a somewhat sparse index.

\paragraph{d)}

How does the system deal with overflows in clustering indexes?
How is the fill factor controlled?

According to \cite{PostgreSQL2024CreateIndex}

\paragraph{e)}

Discuss any further characteristics of the system related to clustering indexes that are relevant to a database tuner.

\youranswerhere{}

\subsection{Non-Clustering Indexes}

Discuss how the system supports non-clustering indexes, in particular:

\paragraph{a)}

How do you create a combined, non-clustering index on \texttt{(dept,salary)}? Show the query.$^1$

\youranswerhere{}

\begin{lstlisting}[style=dbtsql]
[Your SQL query goes here ...]
\end{lstlisting}

\paragraph{b)}

Can the system take advantage of covering indexes? What if the index covers the query, but the condition is not a prefix of the attribute sequence \texttt{(dept,salary)}?

\youranswerhere{}

\paragraph{c)}

Discuss any further characteristics of the system related to non-clustering indexes that are relevant to a database tuner.

\youranswerhere{}

\subsection{Key Compression and Page Size}

If your system supports B$^+$ trees, what kind of key compression (if any) is supported? How large is the default disk page? Can it be changed?

\youranswerhere{}

\subsection*{Time Spent on this Assignment}

Time in hours per person:
\begin{itemize}
  \item Florian Frauenschuh: \textbf{}
  \item Peter Lindner: \textbf{}
  \item Alexander Weilert: \textbf{}
\end{itemize}

\pagebreak

\printbibliography[title=References]

\end{document}
