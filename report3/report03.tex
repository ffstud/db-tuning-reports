\documentclass[11pt]{scrartcl}

\usepackage[top=1.5cm]{geometry}
\usepackage{float}
\usepackage{listings}
\usepackage{xcolor}
\usepackage{csquotes}

\usepackage[
  backend=biber,
  style=numeric,
  sorting=none
]{biblatex}

\addbibresource{references/references.bib}

\setlength{\parindent}{0em}
\setlength{\parskip}{0.5em}

\newcommand{\youranswerhere}{[Your answer goes here \ldots]}
\renewcommand{\thesubsection}{\arabic{subsection}}

\lstdefinestyle{dbtsql}{
  language=SQL,
  basicstyle=\small\ttfamily,
  keywordstyle=\color{magenta!75!black},
  stringstyle=\color{green!50!black},
  showspaces=false,
  showstringspaces=false,
  commentstyle=\color{gray}}

\title{
  \textbf{\large Assignment 3} \\
  Index Tuning \\
  {\large Database Tuning}}

\author{
  New Group 8 \\
  \large Frauenschuh Florian, 12109584 \\
  \large Lindner Peter, 12101607 \\
  \large Weilert Alexander, 12119653
}

\begin{document}

\maketitle

\textbf{Database system and version:} \texttt{Postgres 2.3.4} with driver \texttt{postgresql 42.7.3} \\

\subsection{Index Data Structures}

Which index data structures (e.g., B$^+$ tree index) are supported?

\youranswerhere{}

\subsection{Clustering Indexes}

Discuss how the system supports clustering indexes, in particular:

\paragraph{a)}

How do you create a clustering index on \texttt{ssnum}? Show the query.\footnote{Give the queries for creating a hash index \emph{and} a B$^+$ tree index if both of them are supported.}

\youranswerhere{}

\begin{lstlisting}[style=dbtsql]
[Your SQL query goes here ...]
\end{lstlisting}

\paragraph{b)}

Are clustering indexes on non-key attributes supported, e.g., on \texttt{name}? Show the query.

\youranswerhere{}

\begin{lstlisting}[style=dbtsql]
[Your SQL query goes here ...]
\end{lstlisting}

\paragraph{c)}

Is the clustering index dense or sparse?

\youranswerhere{}

\paragraph{d)}

How does the system deal with overflows in clustering indexes? How is the fill factor controlled?

\youranswerhere{}

\paragraph{e)}

Discuss any further characteristics of the system related to clustering indexes that are relevant to a database tuner.

\youranswerhere{}

\subsection{Non-Clustering Indexes}

Discuss how the system supports non-clustering indexes, in particular:

\paragraph{a)}

How do you create a combined, non-clustering index on \texttt{(dept,salary)}? Show the query.$^1$

\youranswerhere{}

\begin{lstlisting}[style=dbtsql]
[Your SQL query goes here ...]
\end{lstlisting}

\paragraph{b)}

Can the system take advantage of covering indexes? What if the index covers the query, but the condition is not a prefix of the attribute sequence \texttt{(dept,salary)}?

\youranswerhere{}

\paragraph{c)}

Discuss any further characteristics of the system related to non-clustering indexes that are relevant to a database tuner.

\youranswerhere{}

\subsection{Key Compression and Page Size}

If your system supports B$^+$ trees, what kind of key compression (if any) is supported? How large is the default disk page? Can it be changed?

\youranswerhere{}

\subsection*{Time Spent on this Assignment}

Time in hours per person:
\begin{itemize}
  \item Florian Frauenschuh: \textbf{}
  \item Peter Lindner: \textbf{}
  \item Alexander Weilert: \textbf{}
\end{itemize}

\pagebreak

\printbibliography[title=References]

\end{document}
