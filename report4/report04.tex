\documentclass[11pt]{scrartcl}

\usepackage[top=1.5cm]{geometry}
\usepackage{float}
\usepackage{listings}
\usepackage{xcolor}

\setlength{\parindent}{0em}
\setlength{\parskip}{0.5em}

\newcommand{\youranswerhere}{[Your answer goes here \ldots]}
\renewcommand{\thesubsection}{\arabic{subsection}}

\lstdefinestyle{dbtsql}{
  language=SQL,
  basicstyle=\small\ttfamily,
  keywordstyle=\color{magenta!75!black},
  stringstyle=\color{green!50!black},
  showspaces=false,
  showstringspaces=false,
  commentstyle=\color{gray}}

\title{
  \textbf{\large Assignment 4} \\
  Index Tuning -- Selection \\
  {\large Database Tuning}}

\author{
  New Group 8 \\
  \large Frauenschuh Florian, 12109584 \\
  \large Lindner Peter, 12101607 \\
  \large Weilert Alexander, 12119653
}

\begin{document}

\maketitle

\paragraph{Notes}

\begin{itemize}
  \item Do not forget to run \lstinline[style=dbtsql]{ANALYZE tablename} after creating or changing a table.
  \item Use \lstinline[style=dbtsql]{EXPLAIN ANALYZE} for the query plans that you display in the report.
\end{itemize}

\subsection*{Experimental Setup}

How do you send the queries to the database? How do you measure the execution time for a sequence of queries?

\youranswerhere{}

\subsection*{Clustering B$^+$ Tree Index}

\paragraph{Point Query}

Repeat the following query multiple times with different conditions for \texttt{pubID}.

\begin{lstlisting}[style=dbtsql]
SELECT * FROM Publ WHERE pubID = ...
\end{lstlisting}

Which conditions did you use?

\youranswerhere{}

Show the runtime results and compute the throughput.

\youranswerhere{}

Query plan (for one of the queries):

{\small
\parskip0pt\begin{verbatim}
[Your query plan goes here ...]
\end{verbatim}}

\paragraph{Multipoint Query vs. Multipoint Query IN-Predicate -- Low Selectivity}

Repeat the following query multiple times with different conditions for \texttt{booktitle}.

\begin{lstlisting}[style=dbtsql]
SELECT * FROM Publ WHERE booktitle = ...
\end{lstlisting}

\begin{lstlisting}[style=dbtsql]
SELECT * FROM Publ WHERE pubID IN (...)
\end{lstlisting}


Which conditions did you use?

\youranswerhere{}

Show the runtime results and compute the throughput.

\youranswerhere{}

Query plan (for one of the queries):

{\small
\parskip0pt\begin{verbatim}
[Your query plan goes here ...]
\end{verbatim}}

\paragraph{Multipoint Query -- High Selectivity}

Repeat the following query multiple times with different conditions for \texttt{year}.

\begin{lstlisting}[style=dbtsql]
SELECT * FROM Publ WHERE year = ...
\end{lstlisting}

Which conditions did you use?

\youranswerhere{}

Show the runtime results and compute the throughput.

\youranswerhere{}

Query plan (for one of the queries):

{\small
\parskip0pt\begin{verbatim}
[Your query plan goes here ...]
\end{verbatim}}

\subsection*{Non-Clustering B$^+$ Tree Index}

\emph{Note:} Make sure the data is not physically ordered by the indexed attributes due to the clustering index that you created before.

\paragraph{Point Query}

Repeat the following query multiple times with different conditions for \texttt{pubID}.

\begin{lstlisting}[style=dbtsql]
SELECT * FROM Publ WHERE pubID = ...
\end{lstlisting}

Which conditions did you use?

\youranswerhere{}

Show the runtime results and compute the throughput.

\youranswerhere{}

Query plan (for one of the queries):

{\small
\parskip0pt\begin{verbatim}
[Your query plan goes here ...]
\end{verbatim}}

\paragraph{Multipoint Query vs. Multipoint Query IN-Predicate -- Low Selectivity}

Repeat the following query multiple times with different conditions for \texttt{booktitle}.

\begin{lstlisting}[style=dbtsql]
SELECT * FROM Publ WHERE booktitle = ...
\end{lstlisting}

\begin{lstlisting}[style=dbtsql]
SELECT * FROM Publ WHERE pubID IN (...)
\end{lstlisting}


Which conditions did you use?

\youranswerhere{}

Show the runtime results and compute the throughput.

\youranswerhere{}

Query plan (for one of the queries):

{\small
\parskip0pt\begin{verbatim}
[Your query plan goes here ...]
\end{verbatim}}

\paragraph{Multipoint Query -- High Selectivity}

Repeat the following query multiple times with different conditions for \texttt{year}.

\begin{lstlisting}[style=dbtsql]
SELECT * FROM Publ WHERE year = ...
\end{lstlisting}

Which conditions did you use?

\youranswerhere{}

Show the runtime results and compute the throughput.

\youranswerhere{}

Query plan (for one of the queries):

{\small
\parskip0pt\begin{verbatim}
[Your query plan goes here ...]
\end{verbatim}}

\subsection*{Non-Clustering Hash Index}

\emph{Note:} Make sure the data is not physically ordered by the indexed attributes due to the clustering index that you created before.

\paragraph{Point Query}

Repeat the following query multiple times with different conditions for \texttt{pubID}.

\begin{lstlisting}[style=dbtsql]
SELECT * FROM Publ WHERE pubID = ...
\end{lstlisting}

Which conditions did you use?

\youranswerhere{}

Show the runtime results and compute the throughput.

\youranswerhere{}

Query plan (for one of the queries):

{\small
\parskip0pt\begin{verbatim}
[Your query plan goes here ...]
\end{verbatim}}

\paragraph{Multipoint Query vs. Multipoint Query IN-Predicate -- Low Selectivity}

Repeat the following query multiple times with different conditions for \texttt{booktitle}.

\begin{lstlisting}[style=dbtsql]
SELECT * FROM Publ WHERE booktitle = ...
\end{lstlisting}

\begin{lstlisting}[style=dbtsql]
SELECT * FROM Publ WHERE pubID IN (...)
\end{lstlisting}


Which conditions did you use?

\youranswerhere{}

Show the runtime results and compute the throughput.

\youranswerhere{}

Query plan (for one of the queries):

{\small
\parskip0pt\begin{verbatim}
[Your query plan goes here ...]
\end{verbatim}}

\paragraph{Multipoint Query -- High Selectivity}

Repeat the following query multiple times with different conditions for \texttt{year}.

\begin{lstlisting}[style=dbtsql]
SELECT * FROM Publ WHERE year = ...
\end{lstlisting}

Which conditions did you use?

\youranswerhere{}

Show the runtime results and compute the throughput.

\youranswerhere{}

Query plan (for one of the queries):

{\small
\parskip0pt\begin{verbatim}
[Your query plan goes here ...]
\end{verbatim}}

\subsection*{Table Scan}

\emph{Note:} Make sure the data is not physically ordered by the indexed attributes due to the clustering index that you created before.

\paragraph{Point Query}

Repeat the following query multiple times with different conditions for \texttt{pubID}.

\begin{lstlisting}[style=dbtsql]
SELECT * FROM Publ WHERE pubID = ...
\end{lstlisting}

\begin{lstlisting}[style=dbtsql]
SELECT * FROM Publ WHERE pubID IN (...)
\end{lstlisting}


Which conditions did you use?

\youranswerhere{}

Show the runtime results and compute the throughput.

\youranswerhere{}

Query plan (for one of the queries):

{\small
\parskip0pt\begin{verbatim}
[Your query plan goes here ...]
\end{verbatim}}

\paragraph{Multipoint Query vs. Multipoint Query IN-Predicate -- Low Selectivity}

Repeat the following query multiple times with different conditions for \texttt{booktitle}.

\begin{lstlisting}[style=dbtsql]
SELECT * FROM Publ WHERE booktitle = ...
\end{lstlisting}

\begin{lstlisting}[style=dbtsql]
SELECT * FROM Publ WHERE pubID IN (...)
\end{lstlisting}

Which conditions did you use?

\youranswerhere{}

Show the runtime results and compute the throughput.

\youranswerhere{}

Query plan (for one of the queries):

{\small
\parskip0pt\begin{verbatim}
[Your query plan goes here ...]
\end{verbatim}}

\paragraph{Multipoint Query -- High Selectivity}

Repeat the following query multiple times with different conditions for \texttt{year}.

\begin{lstlisting}[style=dbtsql]
SELECT * FROM Publ WHERE year = ...
\end{lstlisting}

Which conditions did you use?

\youranswerhere{}

Show the runtime results and compute the throughput.

\youranswerhere{}

Query plan (for one of the queries):

{\small
\parskip0pt\begin{verbatim}
[Your query plan goes here ...]
\end{verbatim}}

\subsection*{Discussion}

Give the throughput of the query types and index types in queries/second.
\begin{table}[H]
  \centering
  \begin{tabular}{c|c|c|c|c}
    & clustering & non-clust.\ B$^+$ tree & non-clust.\ hash & table scan
      \tabularnewline
    \hline
    point (\texttt{pubID}) & \ldots & \ldots & \ldots & \ldots \tabularnewline
    \hline
    multipoint (\texttt{booktitle}) & \ldots & \ldots & \ldots & \ldots
      \tabularnewline
    \hline
		multipoint-IN (\texttt{pubID}) & \ldots & \ldots & \ldots & \ldots
      \tabularnewline
		\hline
    multipoint (\texttt{year}) & \ldots & \ldots & \ldots & \ldots
      \tabularnewline
  \end{tabular}
\end{table}

Discuss the runtime results for the different index types and the table scan. Are the results expected? Why (not)?

\youranswerhere{}

\subsection*{Time Spent on this Assignment}

Time in hours per person: \textbf{XXX}

\subsection*{References}

\begin{table}[H]
  \centering
  \begin{tabular}{c}
    \hline
    \textbf{Important:} Reference your information sources! \tabularnewline
    Remove this section if you use footnotes to reference your information sources. \tabularnewline
    \hline
  \end{tabular}
\end{table}

\end{document}
