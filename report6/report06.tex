\documentclass[11pt]{scrartcl}

\usepackage[top=1.5cm]{geometry}
\usepackage{url}
\usepackage{hyperref}
\usepackage{float}
\usepackage{listings}
\usepackage{xcolor}
\usepackage{multicol}
\usepackage{booktabs}

\setlength{\parindent}{0em}
\setlength{\parskip}{0.5em}

\newcommand{\youranswerhere}{[Your answer goes here \ldots]}
\renewcommand{\thesubsection}{\arabic{subsection}}

\lstdefinestyle{dbtsql}{
  language=SQL,
  basicstyle=\small\ttfamily,
  keywordstyle=\color{magenta!75!black},
  stringstyle=\color{green!50!black},
  showspaces=false,
  showstringspaces=false,
  commentstyle=\color{gray}}

\title{
  \textbf{\large Assignment 6} \\
  Concurrency Tuning \\
  {\large Database Tuning}}

\author{
  New Group 8 \\
  \large Frauenschuh Florian, 12109584 \\
  \large Lindner Peter, 12101607 \\
  \large Weilert Alexander, 12119653
}

\begin{document}

\maketitle

\paragraph{Notes}

\begin{itemize}
  \item You will need to run transactions concurrently using threads in Java. See \\ \url{https://dbresearch.uni-salzburg.at/teaching/2020ss/dbt/account.zip} for an example.
\end{itemize}

\subsection*{Experimental Setup}

For our experiments we used the following hardware and software:

\begin{multicols}{2}
  \begin{table}[H]
    \centering
    \begin{tabular}{lc}
      \toprule
      Component & Specs \\
      \midrule
      Processor & i7-13700H 3.7-5.0 GHz \\
      Memory & 32 GiB \\
      \bottomrule
    \end{tabular}
    \caption{Hardware: Dell XPS 15 9530}
    \label{table:hardware}
  \end{table}

  \columnbreak

  \begin{table}[H]
    \centering
    \begin{tabular}{lc}
      \toprule
      Software & Version \\
      \midrule
      OS & Ubuntu 22.04 \\
      Postgres & 2.3.4 \\
      postgresql & 42.7.3 \\
      MariaDB & 10.6.16 \\
      mariadb-java-client & 3.3.3 \\
      Java & 18 \\
      \bottomrule
    \end{tabular}
    \caption{Software}
    \label{table:software_versions}
  \end{table}
\end{multicols}

Before each test run we created the table \texttt{accounts} by
\begin{lstlisting}[style=dbtsql]
CREATE TABLE IF NOT EXISTS accounts (
  account INT,
  balance INT
);
\end{lstlisting}

and inserted the initial data by using batch statements.


\subsection*{Solution (a)}

\paragraph{Read Committed}

Throughput and correctness for solution (a) with isolation level \lstinline[style=dbtsql]{READ COMMITTED}.

\begin{table}[H]
  \centering
  \begin{tabular}{c|c|c}
    \#Concurrent Transactions & Throughput [transactions/sec] & Correctness
      \tabularnewline
    \hline
    1 & 714.28 & 1.0 \tabularnewline
    2 & 1123.59 & 0.87 \tabularnewline
    3 & 1176.47 & 0.89 \tabularnewline
    4 & 1020.4 & 0.87 \tabularnewline
    5 & 1315.79 & 0.93 \tabularnewline
  \end{tabular}
\end{table}

\paragraph{Serializable}

Throughput and correctness for solution (a) with isolation level \lstinline[style=dbtsql]{SERIALIZABLE}.

\begin{table}[H]
  \centering
  \begin{tabular}{c|c|c}
    \#Concurrent Transactions & Throughput [transactions/sec] & Correctness
      \tabularnewline
    \hline
    1 & 645.16 & 1.0 \tabularnewline
    2 & 1052.63 & 0.84 \tabularnewline
    3 & 1333.33 & 0.93 \tabularnewline
    4 & 1111.11 & 0.89 \tabularnewline
    5 & 1265.82 & 0.96 \tabularnewline
  \end{tabular}
\end{table}

\subsection*{Solution (b)}

\paragraph{Read Committed}

Throughput and correctness for solution (b) with isolation level \lstinline[style=dbtsql]{READ COMMITTED}.

\begin{table}[H]
  \centering
  \begin{tabular}{c|c|c}
    \#Concurrent Transactions & Throughput [transactions/sec] & Correctness
      \tabularnewline
    \hline
    1 & 917.43 & 1.0 \tabularnewline
    2 & 1219.51 & 1.0 \tabularnewline
    3 & 1315.79 & 1.0 \tabularnewline
    4 & 1333.33 & 1.0 \tabularnewline
    5 & 1408.45 & 1.0 \tabularnewline
  \end{tabular}
\end{table}

\paragraph{Serializable}

Throughput and correctness for solution (b) with isolation level \lstinline[style=dbtsql]{SERIALIZABLE}.

\begin{table}[H]
  \centering
  \begin{tabular}{c|c|c}
    \#Concurrent Transactions & Throughput [transactions/sec] & Correctness
      \tabularnewline
    \hline
    1 & 943.39 & 1.0 \tabularnewline
    2 & 1219.51 & 1.0 \tabularnewline
    3 & 1298.7 & 1.0 \tabularnewline
    4 & 1369.86 & 1.0 \tabularnewline
    5 & 1408.45 & 1.0 \tabularnewline
  \end{tabular}
\end{table}

\subsection*{Discussion}

Discuss the outcomes and explain the difference between the isolation levels in PostgreSQL with respect to your experiment.

Explain \textbf{with your own words} how PostgreSQL deals with updates in the different isolation levels, within a transaction and within a single SQL command. Explicitly explain why you got the experimental results for solution (a) and (b), respectively.

\paragraph{Solution (a)}\mbox{}

\youranswerhere{}

\paragraph{Solution (b)}\mbox{}

\youranswerhere{}

\subsection*{Time Spent on this Assignment}

Time in hours per person:

\begin{itemize}
  \item Florian Frauenschuh: \textbf{x}
  \item Peter Lindner: \textbf{y}
  \item Alexander Weilert: \textbf{z}
\end{itemize}

\subsection*{References}

\begin{table}[H]
  \centering
  \begin{tabular}{c}
    \hline
    \textbf{Important:} Reference your information sources! \tabularnewline
    Remove this section if you use footnotes to reference your information sources. \tabularnewline
    \hline
  \end{tabular}
\end{table}

\end{document}
